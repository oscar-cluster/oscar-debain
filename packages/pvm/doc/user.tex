%
% $Id$
%
% Copyright 2004, Open Cluster Group
% 
% This file is part of the OSCAR distribution from the Open Cluster
% Group.  See the OSCAR and OCG web pages:
% 
% 	http://oscar.sourceforge.net/
% 	http://www.openclustergroup.org/
% 
% The OSCAR framework is covered by the GNU General Public License.  See
% the COPYING file in the top-level OSCAR directory.  Software packages
% that are included in an OSCAR distribution have their own licenses.
% 
% -------------------------------------------------------------------------
%

\subsection{Parallel Virtual Machine (PVM)}
\label{app:pvm-overview}

PVM (Parallel Virtual Machine) is a software package that permits a
heterogeneous collection of Unix and/or Windows computers hooked
together by a network to be used as a single large parallel computer.
Thus large computational problems can be solved more cost effectively
by using the aggregate power and memory of many computers. The
software is very portable. The source, which is available free thru
netlib, has been compiled on everything from laptops to CRAYs.

PVM enables users to exploit their existing computer hardware to solve
much larger problems at minimal additional cost. Hundreds of sites
around the world are using PVM to solve important scientific,
industrial, and medical problems in addition to PVM's use as an
educational tool to teach parallel programming. With tens of thousands
of users, PVM has become the de facto standard for distributed
computing world-wide.

\subsubsection{Using PVM}
\label{app:pvm-usage}

The default OSCAR installation tests PVM via a PBS job (see also:
Section~\ref{app:pbs-overview} on page~\pageref{app:pbs-overview}).
However, some users may choose to use PVM outside of this context so a
few words on usage may be helpful~\footnote{Note, the examples in this
  section assume a shared filesystem, as is used with OSCAR.}.

The default location for user executables is
\file{\$HOME/pvm3/bin/\$PVM\_ARCH}.  On an IA-32 Linux machine, this
is typically of the form: \file{/home/sgrundy/pvm3/bin/LINUX} (replace
``\file{LINUX}'' with ``\file{LINUX64}'' on IA-64).  This is where
binaries should be placed so that PVM can locate them when attempting
to spawn tasks.  This is detailed in the \file{pvm\_intro(1PVM)}
manual page when discussing the environment variables \file{PVM\_PATH}
and \file{PVM\_WD}.

The ``hello world'' example shipped with PVM demonstrates how one can
compile and run a simple application outside of PBS.  The following
screen log highlights this for a standard user \emph{sgrundy} (Solomon
Grundy).

\begin{small}
\begin{verbatim}
    # Crate default directory for PVM binaries (one time operation)
  sgrundy: $ mkdir -p $HOME/pvm3/bin/$PVM_ARCH

    # Copy examples to local 'hello' directory
  sgrundy: $ cp $PVM_ROOT/examples/hello*  $HOME/hello-example
  sgrundy: $ cd $HOME/hello-example

    # Compile a hello world, using necessary include (-I) and library
    # (-L) search path info as well as the PVM3 lib.
  sgrundy: $ gcc -I$PVM_ROOT/include  hello.c -L$PVM_ROOT/lib/$PVM_ARCH  \
  > -lpvm3  -o hello
  sgrundy: $ gcc -I$PVM_ROOT/include  hello_other.c -L$PVM_ROOT/lib/$PVM_ARCH \
  > -lpvm3  -o hello_other

    # Move the companion that will be spawned to the default 
    # PVM searchable directory
  sgrundy: mv hello_other $HOME/pvm3/bin/$PVM_ARCH
\end{verbatim}
\end{small}

\noindent At this point you can start PVM, add hosts to the virtual
machine and run the application:

\begin{small}
\begin{verbatim}
    # Start PVM and add one host "oscarnode1".
  sgrundy: $ pvm
  pvm> add oscarnode1
  add oscarnode1
  1 successful
                     HOST    DTID
               oscarnode1   80000
  pvm> quit
  quit

  Console: exit handler called
  pvmd still running.
  sgrundy: $

    # Run master portion of hello world which contacts the companion.
  sgrundy: $ ./hello
  i'm t40005
  from t80002: hello, world from oscarnode1.localdomain

    # Return to the PVM console and terminate ('halt') the virtual machine.
  sgrundy: $
  sgrundy: $ pvm
  pvmd already running
  pvm> halt
  halt
  Terminated
  sgrundy: $
\end{verbatim}
\end{small}

An alternate method is to use options in the hostfile supplied to
\cmd{pvm} when started from the command-line.  The advantage to the
hostfile options is that you don't have to place your binaries in the
default location or edit any ``.dot'' files.  You can compile and run
the ``hello world'' example in this fashion by using a simple hostfile
as shown here.  

The example below uses the same ``hello world'' program that was
previously compiled, but using a hostfile with the appropriate options
to override the default execution and working directory.  Remember
that the ``\cmd{hello}'' program exists in the
\file{/home/sgrundy/hello-example} directory:

\begin{small}
\begin{verbatim}
  sgrumpy: $ cat myhostfile
  *   ep=/home/sgrundy/hello-example  wd=/home/sgrundy/hello-example
  oscarnode1
\end{verbatim}
\end{small}

\noindent The options used here are: 
\begin{quote}
  \begin{description}
  \item[{\tt *}] -- any node can connect
  \item[{\tt ep}] -- execution path, here set to local directory
  \item[{\tt wd}] -- working directory, here set to local directory
  \item[\emph{nodes}] -- a list of nodes, one per line
  \end{description}
\end{quote}

\noindent Now, we can startup \cmd{pvm} using this \file{myhostfile}
and run the \file{hello} application once again.

\begin{small}
\begin{verbatim}
   # Now, we just pass this as an argument to PVM upon startup.
  sgrundy: $ pvm myhostfile
  pvm> quit
  quit
  
  Console: exit handler called
  pvmd still running.

   # The rest is the same as the previous example
  sgrundy: $ ./hello
  i'm t40005
  from t80002: hello, world from oscarnode1.localdomain

  sgrundy: $ pvm
  pvmd already running
  pvm> halt
  halt
  Terminated
  sgrundy: $
\end{verbatim}
\end{small}

\subsubsection{Other details}

The OSCAR installation of PVM makes use of the \file{env-switcher}
package (also see Section~\ref{app:switcher-overview},
page~\pageref{app:switcher-overview}).  This is where the system-wide
\verb=$PVM_ROOT, $PVM_ARCH= and \verb=$PVM_RSH= environment variable
defaults are set.  Traditionally, this material was included in each
user's ``.dot'' files to ensure availability with non-interactive
shells (e.g. \cmd{rsh}/\cmd{ssh}).  Through the \file{env-ewitcher}
package, a user can avoid any ``.dot'' file adjustments by using the
hostfile directive or default paths for binaries as outlined in the
Usage Section~\ref{app:pvm-usage}.

\noindent For additional information see also:
\begin{itemize}
        \item PVM web site: \url{http://www.csm.ornl.gov/pvm/}
        \item Manual Pages: pvm(1), pvm\_intro(1), pvmd3(1)
        \item Release docs: \verb=$PVM_ROOT/doc/*=
\end{itemize}

